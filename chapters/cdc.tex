\chapter{Cahier des charges}



\section*{Résumé du problème}

Le Bitcoin est la cryptomonnaie la plus connue et une des plus utilisées aujourd'hui en 2021. Cependant, elle n'est pas parfaite et certains points sur son fonctionnement et sa conception peuvent poser problème. Le premier est que la vérification des transactions est très consommatrice d'énergie. A titre d'exemple, la puissance totale de tous les mineurs de Bitcoin regroupés permettrait d'alimenter un pays de taille comparable aux Pays-Bas\cite{BTC_cons}. Le deuxième problème est que les transactions sont consultables publiquement ce qui pose un problème de confidentialité. Il est possible de voir le montant de chaque transaction et ainsi remonter la blockchain pour trouver le solde d'un compte. Cela n'est pas adapter à des transactions plus sensibles comme des versements de salaire par exemple.

\subsection*{Problématique}

En prenant en compte les deux problèmes majeurs du Bitcoin décrits ci-dessus, on peut en déduire que le Bitcoin n'est plus dans l'air du temps même si il est encore très populaire. La problématique est la suivante: il faut trouver une alternative écologique et confidentielle au Bitcoin.

\subsection*{Solutions existantes}

Aujourd'hui en 2021, il existe une multitude de blockchains et cryptomonnaies. Cependant la majorité d'entre elles utilisent le même principe énergivore que Bitcoin ou sont des tokens sur la blockchain Ethereum qui n'est, en 2021, ni plus écologique ni plus confidentielle que Bitcoin.

Mais il y a tout de même des solutions existantes. Il existe des algorithmes beaucoup moins énergivores que la vérification par preuve de travail (Proof-of-Work) utilisé par Bitcoin comme Proof-of-Stake ou encore Proof-of-Space. Concernant la confidentialité, il existes des blockchains confidentielles comme Monero ou Zcash. Cependant, il n'y a pas cryptomonnaie/blockchain qui utilise un algorithme écologie et confidentielle à la fois.

\subsection*{Solutions possibles}

Une solution possible pour résoudre cette problématique est de développer une blockchain qui utilisera un algorithme écologique pour vérifier les transactions comme du Proof-of-Stake ou du Proof-of-Space. Pour les raisons de confidentialité précédemment évoquées, les transactions devront être chiffrées au sein de la blockchain pour pas que leurs montants ou les adresses ne soient consultables publiquement.

\section*{Cahier des charges}

\subsection*{Objectifs}

\subsection*{Besoins}

\subsection*{Contraintes}

\subsection*{Déroulement}

\subsection*{Livrables}
Les délivrables seront les suivants :
\begin{enumerate}
\item Une documentation contenant :
	\begin{itemize}
	\item Une analyse de marché
	\item La décision qui découle de l’analyse
	\item Spécifications
	\item Les informations du module tel que le fonctionnement et les limitations 
	\item Une planification initiale et finale
	\item Un mode d’emploi
	\end{itemize}
\item Un module remplissant les objectifs défini au point 2.1.
\item Un software implémentant les améliorations s’il a été possible de les effectuer.
\end{enumerate}
