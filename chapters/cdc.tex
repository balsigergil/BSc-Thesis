\chapter{Cahier des charges}

\section*{Problématique}

\subsection*{Contexte}

Le Bitcoin est l'une des cryptomonnaies les plus connues et une des plus utilisées aujourd'hui en 2021. Cependant, elle n'est pas parfaite et certains points sur son fonctionnement posent problème. Un des points importants est la vérification des transactions qui est très consommatrice d'énergie. A titre d'exemple, la puissance totale de tous les mineurs de Bitcoin regroupés permettrait d'alimenter un pays de taille comparable aux Pays-Bas \cite{BTC_cons}. Un autre problème notable est que les transactions sont consultables publiquement ce qui pose un problème de confidentialité. Il est possible de voir le montant de chaque transaction et ainsi parcourir la blockchain pour trouver le solde d'un compte. Cela n'est pas adapter à des transactions plus sensibles comme des versements de salaire par exemple.

\subsection*{Solutions existantes}

Il existe une multitude de blockchains et cryptomonnaies. Cependant la majorité d'entre elles utilisent le même principe énergivore que Bitcoin ou sont des tokens sur la blockchain Ethereum qui n'est, en 2021, ni plus écologique ni plus confidentielle que Bitcoin.

Mais il y a tout de même des solutions existantes. Il existe des algorithmes beaucoup moins consommateur d'énergie que la vérification par preuve de travail (Proof-of-Work) utilisée par Bitcoin comme Proof-of-Stake ou encore Proof-of-Space. Concernant les problèmes de confidentialités, il existes des blockchains confidentielles comme Monero ou Zcash. Cependant, il n'y a pas blockchain qui utilise un algorithme écologique et confidentielle à la fois.

\subsection*{Objectif principal}

L'objectif principal dans le but de résoudre cette problématique est de développer une blockchain qui utilisera un protocole de consensus écologique pour vérifier les transactions comme du Proof-of-Stake ou du Proof-of-Space. Pour les raisons de confidentialité précédemment évoquées, les transactions seront chiffrées au sein de la blockchain pour pas que leurs montants ou les adresses ne soient consultables publiquement.

\section*{Cahier des charges}

\subsection*{Objectifs}

\textbf{Travail théorique}
\begin{itemize}
    \item État de l'art des protocoles de consensus avec un poids particulier sur le Proof of Space
    \item État de l'art des applications de preuves à divulgation nulle de connaissance aux blockchains
    \item Explication du fonctionnement d'une cryptomonnaie en général
\end{itemize}

\textbf{Travail pratique}

L'objectif de ce travail pratique est l'implémentation d'une cryptomonnaie en Rust.
Dans un premier temps, l'objectif est d'implémenter un algorithme de consensus par Proof of Space (ou Proof of Space-time) et de réaliser une blockchain utilisant cet algorithme.
Dans un second temps, il sera question d'y intégrer un mécanisme de sécurisation de la blockchain au moyen preuves à divulgation nulle de connaissance afin de rendre les transaction confidentielles.

\subsection*{Si le temps le permet}

\begin{itemize}
    \item Implémentation d'un wallet pour gérer ses transactions
    \item Intégration de smart contracts à la blockchain
\end{itemize}

\subsection*{Livrables}
Les délivrables seront les suivants :
\begin{enumerate}
    \item Une documentation contenant :
    	\begin{itemize}
        	\item Une analyse de l'état de l'art des protocoles de consensus
        	\item Les choix effectués découlant de l'analyse
        	\item Spécifications de la blockchain
    	\end{itemize}
    \item Une cryptomonnaie implémentée en Rust.
\end{enumerate}
