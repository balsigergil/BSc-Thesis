\chapter{Conclusion}

\section{Résultat final}

Ce travail a mis l'accent tout particulièrement sur l'analyse des protocoles de consensus et l'implémentation du \emph{proof of space}. Et c'est dans un deuxième temps, après avoir réalisé une implémentation fonctionnelle du proof of space que la création d'une blockchain a été entreprise. La blockchain développée avait pour objectif de donner un exemple d'utilisation du proof of space avec un cas concret. De plus, sa confection m'a permis d'en savoir plus sur le fonctionnement des cryptomonnaies.

Concernant les protocoles de consensus, il y en a beaucoup qui sont intéressants mais nouveaux et donc n'ont pas énormément été utilisé concrètement. Côté écologie, le proof of work est le pire loin devant les autres mais reste avec la preuve de sécurité la plus fiable. Mais le soucis écologique qu'il pose tend les blockchains à migrer vers d'autres protocoles plus efficace et consommant moins d'énergie. Le proof of space est un protocole tout à fait intéressant qui a suscité pas mal d'intérêt. C'est pourquoi il été choisi pour être implémenté. Le proof of stake est peut-être le deuxième protocole le plus populaire après le proof of work, il est cependant plus difficile à mettre en place sur une nouvelle cryptomonnaie puisque qu'il faut justement miser une partie de cette monnaie. Il reste quand même un bon protocole écologique.

Il est difficile de faire un classement des protocoles étudiés mais on peut dire que proof of space et proof of stake sont deux protocoles répondant aux problèmes écologiques dû au faible besoin de puissance de calcul. Proof of authority sera plus adapté pour des blockchains privées puisqu'il est plus facile de mettre en place des validateurs de confiance dans le secteur privé. Proof of history n'est certes pas un protocole de consensus, il peut s'intégré à un protocole pour permettre un synchronisation rapide comme proof of stake avec la blockchain Solana.

Étant donnée l'importance placée sur le proof of space, c'est concernant ce dernier qu'il y a le plus de résultats. Notamment au niveau des performances où plusieurs tests de benchmark on été réalisé. La conclusion de l'implémentation est qu'il est possible de créer un plot dans le but de trouver des preuves pour un challenge donné. La construction est basée sur la phase 1 du protocole de Chia \cite{chia:construction} et a été finalement entièrement implémentée (la phase 1 pas les autres). La génération du plot final se fait en temps exponentiel par rapport à une constante $k$ alors que la vérification de preuves est très efficace et réalisable en temps constant.

\section{Difficultés rencontrées}

La majeur partie des difficultés rencontrées concerne l'implémentation du proof of space. C'était une construction assez difficile à comprendre au début et il a fallut du temps pour pouvoir l'implémenter. Il a fallut apprendre à utiliser des librairies pour manipuler des bits, ce dont on a pas forcément l'habitude. On travaille le plus souvent en octet. Il a fallut trouver un moyen d'améliorer les performances avec \emph{Rayon}, une librairie de multithreading, pour obtenir un temps d'exécution convenable. Et enfin il a fallut réaliser un moyen de trier les données sur le disque. Parce que comme le protocole génère beaucoup de données, tout ne peut pas être conservé en mémoire, il faut les stocker sur le disque dur. Après la génération, les tables doivent être triée avant de pouvoir générer la suivante. Comme ces tables peuvent atteindre plusieurs giga-octets, il a fallut développer un algorithme de tri externe. C'était une des partie les plus compliquées à faire.

Mais dans l'ensemble il n'y a pas eu de difficulté trop importante qui ait pu bloquer l'avancée du projet. Il a fallut apprendre et comprendre rapidement et il a fallut écrire pas mal de code.


\section{Objectifs non-réalisés}

Mise à part tout le travail réalisé sur le proof of space et la blockchain, il y a quand même des objectifs qui n'ont pas été atteint. Notamment le souhait de créer une cryptomonnaie confidentielle. Cet aspect précis n'as pas été réalisé par manque de temps, l'implémentation du proof of space ayant pris bien 60\% du travail. Il aurait fallut analyser les blockchains \emph{Monero} et \emph{Zcash} pour connaître quel méthodes étaient utilisées pour sécuriser les transactions et voir s'il est possible d'en ajouter une à notre blockchain. Malheureusement cette partie du travail n'a pas pu être faite.

\section{Améliorations possibles du projet}

Il y a une infinité d'améliorations possibles pour le projet mais voici une petite liste non-exhaustive d'améliorations :

\begin{itemize}
  \item Rendre les transactions confidentielles (comme Zcash ou Monero)
  \item Implémenter les phases suivantes du proof of space (nettoyage et compression)
  \item Ajouter un moyen d'avoir un consensus sécurisé (avec des VDF par exemple)
\end{itemize}

\section{Conclusion personnelle}
