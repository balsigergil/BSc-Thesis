\documentclass[../tb_report.tex]{subfiles}

\begin{document}
  
\chapter{Fonctionnement des cryptomonnaies}
\label{ch:presentation}

\section{La blockchain}

...

\section{Les cryptomonnaies}

...

\section{Réseau pair-à-pair distribué}

...

\section{Qu'est-ce qu'un protocole de consensus ?}

Dans une blockchain, un protocole de consensus est un algorithme permettant de mettre l'ensemble des noeuds du réseau d'accord sur une version de la blockchain, ceci en tenant compte du fait que certains noeuds peuvent être malveillants.

Dans une structure centralisée, comme une banque par exemple, les transactions sont vérifiées par la banque elle-même, il est donc difficile de forger de fausses transactions puisque ces dernières sont vérifiées pas une entité centrale. Or, dans une structure décentralisée comme une blockchain, tout le monde peut se joindre au réseau et soumettre des blocs avec des transactions. Certains noeuds peuvent transmettre aux autres noeuds des blocs avec des transactions invalides et commettre des actes frauduleux comme de la double dépense.

Il nous faut donc un algorithme permettant de synchroniser tous les noeuds sur une version identique de la blockchain afin de garantir l'authenticité de tous les blocs qu'elle contient et empêcher qu'une même entité contrôle toute la chaîne de blocs. Ainsi, un protocole de consensus va permettre de déterminer quel noeud va pouvoir effectuer les calcules ou actions nécessaires afin d'ajouter un nouveau bloc à la chaîne. Dans le but de motiver les noeuds à agir de manière honnête, le réseau les récompense le plus souvent lors de la création d'un nouveau bloc avec une certaine quantité de cryptomonnaie. 

Tous les autres noeuds doivent alors se synchroniser et travailler sur la chaîne de blocs la plus longue s'il y a plusieurs branches disponibles. Le but étant que la chaîne de blocs honnête grandisse plus rapidement que d'autres chaînes isolées ou frauduleuses.

\end{document}